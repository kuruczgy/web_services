
% insert template arguments here
%\documentclass{article}
\providecommand \mycount {6}
%\providecommand \mylandscape {}
\providecommand \mycaption {Tesztlekvár}

\ifx \mylandscape \undefined
\usepackage[margin=0cm,a4paper]{geometry}
\else
\usepackage[margin=0cm,a4paper,landscape]{geometry}
\fi
\usepackage{anyfontsize}
\usepackage{tikz}
\setlength{\parindent}{0cm}

\def \nx {1}
\def \ny {\mycount}
\def \notch {1cm}

\begin{document}
\newcommand{\w}{\textwidth}
\newcommand{\h}{\textheight}
\begin{tikzpicture}
\clip (0,0) rectangle (\w,\h);

\ifnum \nx > 1
\foreach \x [parse=true] in {1,...,\nx-1} {
    \def \pos {\w / \nx * \x};
    \draw (\pos, 0) -- (\pos, \h);
}
\fi
\foreach \y [parse=true] in {1,...,\ny-1} {
    \def \pos {\h / \ny * \y};
    \draw (0, \pos) -- (\notch, \pos) (\w-\notch,\pos) -- (\w, \pos);
}

\foreach \x [parse=true] in {0,...,\nx-1} {
    \foreach \y [parse=true] in {0,...,\ny-1} {
        \node at (\w/\nx/2 + \w/\nx*\x, \h/\ny/2 + \h/\ny*\y) {
            \pgfmathparse{\w / \nx}
            \parbox{\pgfmathresult pt}{
                \centering
                {
                    \fontsize{1.1cm}{1cm}\selectfont
                    \mycaption
                } \\[0.3cm]
                { \large \the\year }
            }
        };
    }
}

\end{tikzpicture}
\end{document}
